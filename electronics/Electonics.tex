\documentclass[12pt,titlepage,a4paper]{book}
\usepackage{pgfplots}
\usepackage{siunitx}
\usepackage{amsmath}
\usepackage{siunitx}
\usepackage{mathtools}
\usepackage[thinc]{esdiff}
\usepackage[utf8]{inputenc}
\usepackage[italian]{babel}
\usepackage{gensymb}
\usepackage{bm}
\usepackage{fullpage}

\usepackage{tikz,tikz-3dplot}
\usepackage{amsmath}
\usepackage{mathtools}
\usepackage[thinc]{esdiff}
\usepackage[utf8]{inputenc}
\usepackage[italian]{babel}
\usepackage{textcomp,gensymb}
\usepackage{bm}
\usepackage{calc}

\usepackage{physics}
\usepackage{adjustbox}
\usepackage{balance}
\usepackage{pgfplots}
\usepackage{subfiles}
\usepackage{graphicx}

\usepackage{url}

\usepackage{siunitx}
\DeclareSIUnit\px{px}

\usepackage{pst-optic, pst-optexp}


%%%%%%%%%%%%%%%%%%%%
\newcommand{\der}[3][]{\frac{d^{#1} #2}{d #3^{#1}}}
\newcommand{\dder}[3][]{\frac{\partial{d}^{#1} #2}{\partial{d} #3^{#1} }}
\newcommand{\uvec}[1]{\hat{#1}}

\newcommand{\cvec}[2]{#1\uvec{#2}}
\newcommand{\veccomp}[4][]{\cvec{#2}{i^{#1}} + \cvec{#3}{j^{#1}} + \cvec{#4}{k^{#1}}}

\newcommand{\phasor}[1]{\mathbf{#1}}
%%%%%%%%%%%%%%%%%%%%


\pgfplotsset{compat = newest}


\title{Cheat Sheet}
\author{Michele Forese\\
  \small{\texttt{michele.forese.personal@gmail.com}}
}
\date{\today}

\begin{document}

\maketitle

% -- Bibliography (APA style)
\bibliography{references}

\pagebreak

\section{Potenza in regine Sinusoidale}
\subsection{Potenza Istantatnea e potenza media}


$$ v(t) = V_m cos(\omega t + \theta_v) $$
$$ i(t) = I_m cos(\omega t + \theta_i) $$

La potenza istantanea è:

\begin{equation*}
  p(t) = V_m cos(\omega t + \theta_v) I_m cos(\omega t + \theta_i) [\SI{}{\watt}]
\end{equation*}

Applicando le formule di Werner della trigonometria,
$$ cos(A)cos(B) = \frac{1}{2} [cos(A - B) + cos(A + B)] $$

Possiamo riscrivere la formula:

$$ p(t)=\frac{1}{2}V_mI_mcos(\theta_v - \theta_i) + \frac{1}{2}V_mI_mcos(2 \omega t + \theta_v + \theta_i) $$

\paragraph{Potenza Attiva}

$$ P = \frac{1}{2} V_m I_m cos(\theta_v - \theta_i) $$

\begin{center}
  \begin{tikzpicture}
    \begin{axis}[
        domain = 0:10,
      ]
      \addplot[
        samples = 200,
        smooth,
        thick,
        blue,
      ]{ 0.5 * cos(deg(20)) + (1/2) * cos(deg(2 * x + 90))};
      \addplot[
        red,
        thin,
        dashed
      ]{0.5 * cos(deg(20))};
    \end{axis}
  \end{tikzpicture}
\end{center}

Ampiezza oscillazione: $ A = \frac{1}{2} V_m I_m $

Picco massimo: $ P_{max} = P + \frac{1}{2} V_m I_m $

Media Potenza in $ T = \frac{2 \pi}{\theta} $

\begin{equation*}
  \begin{split}
    P_{avg} & = \frac{1}{T} \int_{0}^{T} p(t) dt \\
    & = \frac{1}{T} \int_{0}^{T} \frac{1}{2} V_m I_m cos(\theta_v - \theta_i) dt +
    \frac{1}{T} \int_{0}^{T} \frac{1}{2}V_mI_mcos(2 \omega t + \theta_v + \theta_i) dt = \\
    & = \frac{1}{2} V_m I_m cos(\theta_v - \theta_i) + 0 = P
  \end{split}
\end{equation*}

Potenza Attiva:

Valore medio di un periodo della potenza instantanea $p(t)$

Energia assorbita:

\begin{equation*}
  w = \int_{0}^{\Delta t} p(t) dt = P \Delta t + \int_{0}^{\Delta t} \frac{1}{2} V_m I_m cos(2 \omega t + \theta_v + \theta_i) dt
\end{equation*}

\subsubsection{Resistore}

\begin{equation*}
  i(t) = I_m cos(\omega t + \theta_i) \\
  v(t) = Ri(t) = R I_m cos(\omega t + \theta_i)
\end{equation*}

Dunque: $\theta_v = \theta_i \Rightarrow cos(\theta_v - \theta_i) = 1$

$$ V_m = R I_m $$

\begin{equation*}
  p(t) = \frac{1}{2}V_mI_mcos(\theta_v - \theta_i) + \frac{1}{2}V_mI_mcos(2 \omega t + \theta_v + \theta_i)
\end{equation*}

Diventa:

\begin{equation*}
  p(t) = \frac{1}{2} R {I_m}^2 + \frac{1}{2} R {I_m}^2 cos(2 \omega t + 2 \theta_i)
\end{equation*}

\begin{list}{-}{La potenza istantanea}
  \item è sempre non negativa
  \item sia annulla con periodo $ \frac{T}{2}$
\end{list}

Potenza media:

\begin{equation*}
  P = \frac{1}{2} V_m I_m = \frac{1}{2} R {I_m}^2 = \frac{1}{2} \frac{{V_m}^2}{R}
\end{equation*}

Potenza di Picco (max):

\begin{equation*}
  P_{max} = R {I_m}^2 = \frac{{V_m}^2}{R}
\end{equation*}

\subsubsection{Induttore}


\begin{equation*}
  \begin{split}
    i(t) &= I_m cos(\omega t + \theta_i) \\
    v(t) &= L \diff{i}{t} = - \omega L I_m sen(\omega t + \theta_i) = \\
    &= \omega L I_m cos(\omega t + \theta_i + 90\degree)
  \end{split}
\end{equation*}


Dunque:

$$\theta_v = \theta_i + 90\degree \Rightarrow cos(\theta_v - \theta_i) = 0$$
$$ V_m = \omega L I_m $$

\begin{equation*}
  p(t) = - \frac{1}{2} \omega L {I_m}^2 sen(\theta_v - \theta_i) + \frac{1}{2}V_mI_mcos(2 \omega t + \theta_v + \theta_i)
\end{equation*}

Diventa:

\begin{equation*}
  p(t) = - \frac{1}{2} \omega L {I_m}^2 sen(2 \omega t + 2 \theta_i)
\end{equation*}

\begin{list}{-}{La potenza istantanea}
  \item sia annulla con periodo $ \frac{T}{4}$
  \item Assorbe energia in un intervallo di tempo pari a $\frac{T}{4} (p(t) > 0)$ e
        la restituisce nel quarto di periodo successivo
\end{list}

Potenza media:

\begin{equation*}
  P = 0
\end{equation*}

Potenza di Picco (max):

\begin{equation*}
  P_{max} = \frac{1}{2} \omega L {I_m}^2
\end{equation*}


\subsubsection{Condensatore}


\begin{equation*}
  \begin{split}
    v(t) &= V_m cos(\omega t + \theta_v) \\
    i(t) &= C \diff{v}{t} = - \omega C V_m sen(\omega t + \theta_v) = \\
    &= \omega C V_m cos(\omega t + \theta_v + 90\degree)
  \end{split}
\end{equation*}


Dunque:

$$\theta_v = \theta_i + 90\degree \Rightarrow cos(\theta_v - \theta_i) = 0$$
$$ I_m = \omega C V_m $$


Diventa:

\begin{equation*}
  p(t) = \frac{1}{2} \omega C {V_m}^2 sen(2 \omega t + 2 \theta_i)
\end{equation*}

\begin{list}{-}{La potenza istantanea}
  \item sia annulla con periodo $ \frac{T}{4}$
  \item Assorbe energia in un intervallo di tempo pari a $\frac{T}{4} (p(t) > 0)$ e
        la restituisce nel quarto di periodo successivo
\end{list}

Potenza media:

\begin{equation*}
  P = 0
\end{equation*}

Potenza di Picco (max):

\begin{equation*}
  P_{max} = \frac{1}{2} \omega C {V_m}^2
\end{equation*}

\subsection{Valore Efficace}

$$ I_{eff} = \frac{I_m}{\sqrt{2}} $$

Il valore efficace di una corrente sinusoidale è il valore di quella tensione costante che,
scorrendo nello stesso resistore, provoca la dissipazione della stessa potenza media

$$ V_{eff} = \frac{V_m}{\sqrt{2}} $$

Il valore efficace di una tensione sinusoidale è il valore di quella tensione costante che,
ai capi dello stesso resistore, provoca la dissipazione della stessa potenza media

Legge di Ohm per i valori efficaci:

$$ V_{eff} = R I_{eff} $$

\begin{equation}
  I_{eff} = \sqrt{\frac{1}{T} \int_{0}^{T} i^2(t) dt}
\end{equation}

\subsection{Potenza complessa}

\begin{equation}
  \pmb{S} = \frac{1}{2} \pmb{V} \pmb{I^{*}}
\end{equation}


\subfile{./electronics/trasformatore.tex}

\end{document}