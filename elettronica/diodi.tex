\chapter{Diodi}

\section{Diodo a giunzione}

\begin{equation}
  I_A = I_s \left(e^{\frac{e}{aKT} V_{AK}} - 1\right)
\end{equation}

dove:

\begin{conditions}
  I_s & corrente di saturazione inversa del diodo \\
  e & carica dell'elettrone \\
  T & temperatura assoluta in gradi Kelvin \\
  K & costante di Boltzmann ($1,38 \cdot 10^{-23} \si[]{\joule/\kelvin}$)\\
  a & coefficiente, (germanio $\cong$ 1, silicio $\cong$ 2)
\end{conditions}

Semplificando:

\begin{equation}
  \begin{split}
    I_a \cong I_{s} e^{\frac{e}{aKT} V_{AK}} \\
    m = \frac{e}{aKT} I_{A}
  \end{split}
\end{equation}

\paragraph{Potenza dissipata}

\begin{equation}
  P_{d} = \frac{T_{j} - T_{a}}{\theta}
\end{equation}

\section{Diodo a punta di contatto}

\section{Serie di un diodo con una resistenza e una f.e.m. continua}

\begin{equation}
  I_{A} = f(V_{AK})
\end{equation}

\begin{equation}
  V_{AK} = E_{A} - R I_{A}
\end{equation}

\begin{equation}
  I_{A} = \frac{E_{A} - V_{AK}}{R}
\end{equation}

\section{Diodo con in serie una resistenza e una f.e.m. alternata}

Tensione ai capi della resistenza $R$

\begin{equation}
  V_{RM} = E_{M} - V_{S}
\end{equation}

Tensione Diodo

\begin{equation}
  v_{diodo} = e - R I_{A}
\end{equation}

Valore Massimo corrente

\begin{equation}
  I_{AM} = \frac{E_{M} - V_{AK}}{R}
\end{equation}

Valore medio della corrente $i_{A}$

\begin{equation}
  I_{m} = \frac{I_{AM}}{\pi}
\end{equation}

Valore medio $V_{Rm}$ della tensione ai capi della resistenza $R$

\begin{equation}
  V_{Rm} = \frac{V_{RM}}{\pi} = \frac{R_{M} - V_{S}}{\pi}
\end{equation}

\section{Diodo con in serie una resistenza, una f.e.m. continua e alternata}

\begin{equation}
  e = E_{M} \cos(\omega t)
\end{equation}

\subsection{$E_{M} < E_{A}$}

\begin{equation}
  i_{At} = I_{AO} + i_{A} = \frac{E_{A}}{R} + \frac{E_{M}}{R} \cos(\omega t)
\end{equation}

\subsection{$E_{M} > E_{A}$}

\begin{equation}
  E_{A} + E_{M} \cos(\theta) = 0
\end{equation}

si deduce che:

\begin{equation}
  \theta = \arccos(-\frac{E_{A}}{E_{M}})
\end{equation}

\paragraph{Angolo di circolazione}

\begin{equation}
  2 \theta = 2 \arccos(-\frac{E_{A}}{E_{M}})
\end{equation}
