\chapter{Trasformatori}

\subsection{Autotrasformatore Ideale}

\begin{equation}
  \begin{split}
    v_3 = N_1 \der{\varPhi}{t} \\
    v_1 = N_2 \der{\varPhi}{t}
  \end{split}
\end{equation}

quindi:

\begin{equation}
  v_3 = \frac{N_1}{N_2} v_1
\end{equation}

Per la LKT,

\begin{equation}
  v_2 = v_3 + v_1
\end{equation}

\begin{equation}
  \boxed{
    v_2 = \frac{N_1 + N_2}{N_2} v_1
  }
\end{equation}

Applicando la legge di Ampere ad una linea chiuda concatenata con l'avvolgimento:

\begin{equation}
  i_3 = \frac{N_1}{N_2} i_2
\end{equation}

\begin{equation}
  \boxed{i_2 = - \frac{N_2}{N_1 + N_2} i_1}
\end{equation}

\paragraph{rapporto spire}

\begin{equation}
  n = \frac{N_2}{N_1 + N_2}
\end{equation}

\paragraph{Abbassamento di tensione}

\begin{align*}
  v_2 = \frac{N_2}{N_1 + N_2} v_1
   & i_2 = - \frac{N_1 + N_2}{N_2} i_2 \\
\end{align*}

%%%%%%%%%%%%%%%%%%%%%%%%%%%%%%%%%%%%%%%%%%%
\section{Induttori Accoppiati}

\subsection{Analisi di circuiti con induttori accoppiati}

\paragraph{Coefficiente di accoppiamento}

\begin{equation}
  k = \frac{M}{\sqrt{L_1 L_2}}
\end{equation}

\paragraph{Potenza Complessa}

\begin{equation}
  \begin{split}
    \phasor{S} &= \frac{1}{2} \left(\phasor{V_1} \phasor{I_1^*} + \phasor{V_2} \phasor{I_2^*}\right) = \\
    &= \frac{1}{2} jw L_1 I_1^2 + \frac{1}{2} jw L_2 I_2^2 + jw M I_1 I_2 \cos(\theta_1 - \theta_2)
  \end{split}
\end{equation}

\paragraph{Regime Sinusoidale}

\begin{equation}
  \begin{split}
    \phasor{V_1} &= j w L_1 \phasor{I_1} + j w M \phasor{I_2} \\
    \phasor{V_2} &= j w M \phasor{I_1} + j w L_2 \phasor{I_2}
  \end{split}
\end{equation}


In forma matriciale:

\begin{equation}
  \begin{bmatrix}
    \phasor{V_1} \\
    \phasor{V_2}
  \end{bmatrix}
  = jw
  \begin{bmatrix}
    L_1 & M   \\
    M   & L_2
  \end{bmatrix}
  \begin{bmatrix}
    \phasor{I_1} \\
    \phasor{I_2}
  \end{bmatrix}
\end{equation}

dove

\begin{equation}
  \phasor{L} =
  \begin{bmatrix}
    L_1 & M   \\
    M   & L_2
  \end{bmatrix}
\end{equation}

è detta \textbf{matrice delle induttanze}.

Può essere anche invertita, fornendo l'espressione delle correnti din funzione delle tensioni:

\begin{equation}
  \begin{bmatrix}
    \phasor{I_1} \\
    \phasor{I_2}
  \end{bmatrix}
  = \frac{1}{jw \Delta}
  \begin{bmatrix}
    L_2 & -M  \\
    -M  & L_1
  \end{bmatrix}
  \begin{bmatrix}
    \phasor{V_1} \\
    \phasor{V_2}
  \end{bmatrix}
\end{equation}

$\Delta$ è il \textbf{determinante della matrice delle induttanze},

\begin{equation}
  \Delta = L_1 L_2 - M^2 = L_1 L_2 \left(1 - k^2\right)
\end{equation}


