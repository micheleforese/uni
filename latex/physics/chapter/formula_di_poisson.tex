\section{Formula di Poisson}

\paragraph{Direzione e verso}

Quando $\Delta t \rightarrow 0$ anche l'angolo $\Delta\phi \rightarrow 0$.
In Questo limite $\alpha = \beta \rightarrow \frac{\pi}{2}$; quindi il vettore $\Delta\vec{a}$ tende
ad essere ortogonale al vettore $\vec{a}$:

\begin{equation}
  \Delta \vec{a} \perp \vec{a} \\
  \diff{\vec{a}}{t} \perp \vec{a}
\end{equation}

Quindi il vettore \input{figures/physics/etc/diff_vec_a_t.tex}
ha direzione ortogonale a quella di $\vec{a}$
e verso indicato dal senso di rotazione del vettore.

\paragraph{Modulo}

Calcoliamo li modulo del del vettore \input{figures/physics/etc/diff_vec_a_t.tex}

\begin{equation}
  \begin{split}
    \absolutevalue{\diff{\vec{a}}{t}}
    &= \lim_{\Delta t \to 0} \absolutevalue{\frac{\Delta \vec{a}}{\Delta t}}
    = \lim_{\Delta t \to 0} \absolutevalue{\Delta \vec{a}} \\
    \absolutevalue{\Delta \vec{a}} &= 2 \cdot a \cdot \sin(\frac{\Delta \phi}{2}) \\
    \absolutevalue{\diff{\vec{a}}{t}} &= \lim_{\Delta t \to 0, \Delta\phi \to 0} {
      \frac{2 \cdot a \cdot \sin(\frac{\Delta \phi}{2})}
      {\Delta t}
    } \\
    &= a \cdot \lim_{\Delta t \to 0, \Delta\phi \to 0} {
      2 \cdot
      {
        \frac{\sin(\frac{\Delta \phi}{2})}
        {\frac{\Delta\phi}{2}}
      }
      \cdot \frac{\Delta\phi}{2}
      \cdot \frac{1}{\Delta t}
    } \\
    &= a \cdot \lim_{\Delta\phi \to 0} {
      \frac{\sin(\frac{\Delta \phi}{2})}
      {\frac{\Delta\phi}{2}}
    }
    \cdot \lim_{\Delta t \to 0} {\frac{\Delta\phi}{\Delta t}}
    = a \cdot \diff{\phi}{t}
    = a \cdot w
  \end{split}
\end{equation}

dove $w = \diff{\phi}{t}$ è il modulato della velocità angolare del vettore rotante.

Possiamo riunire le relazioni ottenute in un'unica formula:

\begin{equation}
  \diff{\vec{a}}{t} = \vec{w} \crossproduct \vec{a} \qquad \text{se} \ \absolutevalue{\vec{a}} = cost.
  \quad \text{(Formula di Poisson)}
\end{equation}

dove $\crossproduct$ indica un prodotto vettoriale ed il vettore $\vec{w} = \vec{w}(t)$ è cosi definito:

\begin{description}
  \item[direzione]: ortogonale al piano di rotazione istantanea di $\vec{a}$;
  \item[verso]: tale che punti verso un osservatore che vede ruotare il vettore $\vec{a}$ in senso antiorario;
  \item[modulo]: $w = \diff{\phi}{t}$ dove $\, \phi \,$ è l'angolo spazzato dal vettore durante la rotazione.
\end{description}


