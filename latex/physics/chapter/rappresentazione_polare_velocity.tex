\chapter{Moto Piano Vettoriale}

\section{Rappresentazione Polare della Velocità}

La posizione di un punto nel piano può essere individuata utilizzando un sistema di coordinate diverse da quelle cartesiane
, per esempio quelle polari.
In particolare possiamo ottenere le rappresentazioni polari del vettore posizione e del vettore velocità in
termini della coppia di coordinate polari $(r, \varphi)$ formata dal modulo del vettore $r$ e dell'angolo
che quello forma con una direzione assegnata (in questo caso l'asse delle x).
A tale scopo è necessario introdurre una nuova coppia di vettori ortogonali. che indicheremo rispettivamente
con i simboli $\hat{r}$ e $\hat{\varphi}$ ottenuta ruotando la coppia di versioni $\hat{i}$ e $\hat{j}$
dell'angolo $\varphi$.


\begin{equation}
  \begin{cases}
    x = r \, cos(\varphi) \\
    y = r \, sen(\varphi)
  \end{cases}
\end{equation}

è Chiaro altresì che, a differenza della coppia di versori
$\lbrace \hat{i}, \hat{j} \rbrace$, la coppia
$\lbrace \hat{r}, \hat{\varphi} \rbrace$ non è costante nel
tempo, bensì la sua orientazione varia al variare della posizione
del punto durante il moto.
Ricordiamoci questa formula (), decidiamo a scrivere il vettore tramite
il prodotto del modulo e del versore ad esso associato.

\begin{equation}
  \vec{r} = r \cdot \hat{r}
\end{equation}


Nel caso particolare di un moto circolare sappiamo che il modulo
$r$ rimane costante e pari al raggio della traiettoria,
invertire in generale $r$ varierà nel tempo e la velocità
$\vec{v}$ si otterrà derivando rispetto al tempo agli entrambi i membri


\begin{equation}
  \vec{v} = \vec{\dot{r}} = \dot{r} \, \hat{r} + r \, \dot{\hat{r}}
\end{equation}

\paragraph{Formula di Poisson}

$$ \diff{u}{t} = \diff{s}{t} \hat{u_N}$$

$$ \diff{s}{t} = \diff{(|\hat{u}| \theta)}{t} \hat{u_N} = |\hat{u}| \diff{\theta}{t} \hat{u_N} $$

\begin{equation}
  \diff{\hat{r}}{t} = \vec{w} \times \hat{r}
  = \dot{\varphi} \, \hat{k} \times \hat{r}
  = \dot{\varphi} \hat{\varphi}
\end{equation}

$ \hat{k} \times \hat{r} = \hat{\varphi} $ Significa ruotarlo di
$90\degree$ e mi ritrovo $ \hat{\varphi}$


\begin{figure}
  \centering
  \resizebox{.5\textwidth}{!}{\subfile{../figures/physics/polar_velocity.tikz}}
  \caption{This is the Polar Representation of the unit vector}
\end{figure}


Rappresentazione Polare della velocità:
\begin{equation}
  \vec{v} =(\dot{r}) \, \hat{r}
  + (r \, \dot{\varphi}) \, \hat{\varphi}
\end{equation}

\begin{list}{-}{Componenti:}
  \item $(\dot{r})$ : Componente radiale $V_r$
  \item $(r \, \dot{\varphi})$ : Componente azimutale $V_{\varphi}$
\end{list}

\paragraph{Rappresentazione intrinseca della velocità}

$$ \vec{v} = \dot{s} \cdot \hat{\tau} $$

\paragraph{Componenti polari della velocità}

\begin{equation}
  \vec{v} = \diff{\vec{r}}{t} = \diff{r}{t} \hat{u_r} + r \diff{\hat{u_r}}{t}
  \rightarrow \vec{v} = \diff{r}{t} \hat{u_r} + r \diff{\theta}{t} \hat{u_{\theta}}
\end{equation}