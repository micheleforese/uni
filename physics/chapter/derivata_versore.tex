\section{Derivata di un Versore}

\begin{equation}
  \Delta \vec{r} = \vec{r}(t + \Delta t) - r(t)
\end{equation}

con $\Delta t \rightarrow t$

$$ \vec{r} = \abs*{r} \hat{u_r} $$

\begin{equation}
  \diff{\vec{r}}{t} = \diff{(\abs*{r} \hat{u_r})}{t} = \diff{r}{t} \hat{u_r} + r \diff{\hat{u_r}}{t}
\end{equation}

Derivata di un Versore:

\begin{equation}
  \begin{split}
    \diff{(\vec{u} \cdot \vec{u})}{t} &= \diff{(\abs*{u} \abs*{u} \cos(\alpha))}{t} \\
    & = \diff{(\abs*{u} \abs*{u})}{t} = 0
  \end{split}
\end{equation}

con $\alpha = $ angolo tra $\vec{u}$ e $\vec{u}$ $(=0)$

Ora calcoliamo la derivata con le regole di derivazione

\begin{equation}
  \diff{(\vec{u} \cdot \vec{u})}{t} = \diff{\vec{u}}{t} \cdot \hat{u} + \vec{u} \cdot \diff{\vec{u}}{t}
  = 2 \vec{u} \cdot \diff{\vec{u}}{t}
\end{equation}

Ciao

\begin{equation}
  2 \vec{u} \cdot \diff{\vec{u}}{t} = 0
\end{equation}

Questo significa che

\begin{equation}
  \vec{u} \perp \diff{\vec{u}}{t}
\end{equation}

La derivata di un vettore è perpendicolare al vettore stesso.

Questo ovviamente vale anche per i versori, che non sono altro che vettori unitari,

Secondo le Formule di Poisson:

\begin{equation}
  \diff{\hat{u}}{t} = \vec{w} \crossproduct \hat{u} = w \, \hat{u_k} \crossproduct \hat{u}
\end{equation}

Il versore della velocità angolare $(\hat{u_k})$ deve per forza assumere questa posizione per l'equazione ()

La velocità angolare è definita:

\begin{equation}
  \vec{w} = \dot{\theta} \, \hat{u_k}
\end{equation}

quindi:

\begin{equation}
  \diff{\hat{u}}{t} = \dot{\theta} \, \hat{u_k} \crossproduct \hat{u}
\end{equation}

Il versore della derivata dell'angolo $ \hat{u_{\theta}} =\hat{u_k} \crossproduct \hat{u} $
ha lo stesso verso della derivata del versore

\begin{equation}
  \diff{\hat{u}}{t} \parallel \hat{u_{\theta}}
\end{equation}

Secondo l'equazione ()

\begin{equation}
  \diff{\vec{r}}{t} = \diff{r}{t}\hat{u_r} + r \diff{\hat{u_r}}{t} = \diff{r}{t}\hat{u_r} + r \diff{\theta}{t}\hat{u_{\theta}}
\end{equation}

$\dot{\vec{r}}$ è una misura di velocità, quindi i termini di () devono essere una somma di velocità.


\begin{description}
  \item [$\diff{r}{t} = v_r$]: è diretta esternamente
  \item [$r \diff{\theta}{t} = v_{\theta}$]: Ha direzione Perpendicolare
\end{description}

\begin{equation}
  \diff{\vec{r}}{t} = \vec{v_r} + \vec{v_{\theta}} = v_r \hat{u_r} + v_{\theta} \hat{u_{\theta}}
\end{equation}
